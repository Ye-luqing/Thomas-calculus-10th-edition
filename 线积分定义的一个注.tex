\documentclass[a4paper, 12pt]{article} % Font size (can be 10pt, 11pt or 12pt) and paper size (remove a4paper for US letter paper)
\usepackage{amsmath,amsfonts,bm}
\usepackage{hyperref,verbatim}
\usepackage{amsthm,epigraph} 
\usepackage{amssymb}
\usepackage{framed,mdframed}
\usepackage{graphicx,color} 
\usepackage{mathrsfs,xcolor} 
\usepackage[all]{xy}
\usepackage{fancybox} 
 \usepackage{xeCJK}

\newtheorem*{adtheorem}{定理}
 \setCJKmainfont[BoldFont=FZYaoTi,ItalicFont=FZYaoTi]{FZYaoTi}
\definecolor{shadecolor}{rgb}{1.0,0.9,0.9} %背景色为浅红色
\newenvironment{theorem}
{\bigskip\begin{mdframed}[backgroundcolor=gray!40,rightline=false,leftline=false,topline=false,bottomline=false]\begin{adtheorem}}
    {\end{adtheorem}\end{mdframed}\bigskip}
\newtheorem*{bdtheorem}{定义}
\newenvironment{definition}
{\bigskip\begin{mdframed}[backgroundcolor=gray!40,rightline=false,leftline=false,topline=false,bottomline=false]\begin{bdtheorem}}
    {\end{bdtheorem}\end{mdframed}\bigskip}
\newtheorem*{cdtheorem}{习题}
\newenvironment{exercise}
{\bigskip\begin{mdframed}[backgroundcolor=gray!40,rightline=false,leftline=false,topline=false,bottomline=false]\begin{cdtheorem}}
    {\end{cdtheorem}\end{mdframed}\bigskip}
\newtheorem*{ddtheorem}{注}
\newenvironment{remark}
{\bigskip\begin{mdframed}[backgroundcolor=gray!40,rightline=false,leftline=false,topline=false,bottomline=false]\begin{ddtheorem}}
    {\end{ddtheorem}\end{mdframed}\bigskip}
\newtheorem*{edtheorem}{引理}
\newenvironment{lemma}
{\bigskip\begin{mdframed}[backgroundcolor=gray!40,rightline=false,leftline=false,topline=false,bottomline=false]\begin{edtheorem}}
    {\end{edtheorem}\end{mdframed}\bigskip}
\newtheorem*{pdtheorem}{例}
\newenvironment{example}
{\bigskip\begin{mdframed}[backgroundcolor=gray!40,rightline=false,leftline=false,topline=false,bottomline=false]\begin{pdtheorem}}
    {\end{pdtheorem}\end{mdframed}\bigskip}

\usepackage[protrusion=true,expansion=true]{microtype} % Better typography
\usepackage{wrapfig} % Allows in-line images
\usepackage{mathpazo} % Use the Palatino font
\usepackage[T1]{fontenc} % Required for accented characters
\linespread{1.05} % Change line spacing here, Palatino benefits from a slight increase by default

\makeatletter
\renewcommand\@biblabel[1]{\textbf{#1.}} % Change the square brackets for each bibliography item from '[1]' to '1.'
\renewcommand{\@listI}{\itemsep=0pt} % Reduce the space between items in the itemize and enumerate environments and the bibliography

\renewcommand{\maketitle}{ % Customize the title - do not edit title
  % and author name here, see the TITLE block
  % below
  \renewcommand\refname{参考文献}
  \newcommand{\D}{\displaystyle}\newcommand{\ri}{\Rightarrow}
  \newcommand{\ds}{\displaystyle} \renewcommand{\ni}{\noindent}
  \newcommand{\pa}{\partial} \newcommand{\Om}{\Omega}
  \newcommand{\om}{\omega} \newcommand{\sik}{\sum_{i=1}^k}
  \newcommand{\vov}{\Vert\omega\Vert} \newcommand{\Umy}{U_{\mu_i,y^i}}
  \newcommand{\lamns}{\lambda_n^{^{\scriptstyle\sigma}}}
  \newcommand{\chiomn}{\chi_{_{\Omega_n}}}
  \newcommand{\ullim}{\underline{\lim}} \newcommand{\bsy}{\boldsymbol}
  \newcommand{\mvb}{\mathversion{bold}} \newcommand{\la}{\lambda}
  \newcommand{\La}{\Lambda} \newcommand{\va}{\varepsilon}
  \newcommand{\be}{\beta} \newcommand{\al}{\alpha}
  \newcommand{\dis}{\displaystyle} \newcommand{\R}{{\mathbb R}}
  \newcommand{\N}{{\mathbb N}} \newcommand{\cF}{{\mathcal F}}
  \newcommand{\gB}{{\mathfrak B}} \newcommand{\eps}{\epsilon}
  \begin{flushright} % Right align
    {\LARGE\@title} % Increase the font size of the title
    
    \vspace{50pt} % Some vertical space between the title and author name
    
    {\large\@author} % Author name
    \\\@date % Date
    
    \vspace{40pt} % Some vertical space between the author block and abstract
  \end{flushright}
}

% ----------------------------------------------------------------------------------------
%	TITLE
% ----------------------------------------------------------------------------------------
\begin{document}
\title{\textbf{线积分定义的一个注}}
% \setlength\epigraphwidth{0.7\linewidth}
\author{\small{叶卢庆}\\{\small{杭州师范大学理学院,学
      号:1002011005}}\\{\small{Email:h5411167@gmail.com}}} % Institution
\renewcommand{\today}{\number\year. \number\month. \number\day}
\date{\today} % Date
% ----------------------------------------------------------------------------------------
  
  
\maketitle % Print the title section
  
% ----------------------------------------------------------------------------------------
% ABSTRACT AND KEYWORDS
% ----------------------------------------------------------------------------------------
  
% \renewcommand{\abstractname}{摘要} % Uncomment to change the name of the abstract to something else
  
% \begin{abstract}
  
% \end{abstract}
  
% \hspace*{3,6mm}\textit{关键词:} % Keywords
  
% \vspace{30pt} % Some vertical space between the abstract and first section
  
% ----------------------------------------------------------------------------------------
% ESSAY BODY
% ----------------------------------------------------------------------------------------
设函数 $f(x,y,z)$ 在光滑曲线(导函数连续) $l$ 上有定义且连续.$l$ 的方程
为
$$
\begin{cases}
  x=x(t),\\
  y=y(t),\\
  z=z(t)\\
\end{cases}(t_0\leq t\leq T)
$$
则
$$
\int_lf(x,y,z)ds=\int_{t_0}^Tf[(x(t),y(t),z(t)]\sqrt{x_t'^2+y_t'^2+z_t'^2}dt.
$$
\begin{remark}
  该定理有明显的物理意义.我们发现 $\sqrt{x_t'^2+y_t'^2+z_t'^2}$ 就是质
  点在 $t$ 时刻的运动速度.我认为该定理本质上就是变量替换公式在发挥作用.
\end{remark}
\begin{proof}[证明]
对时间区间 $[t_0,T]$ 进行 $n$ 等分,成为
$$[t_0,t_0+\frac{T-t_0}{n}],\cdots,[t_0+(n-1)\frac{T-t_0}{n},t_0+n
\frac{T-t_0}{n}]$$
在第 $i$ 个时间区间上,任选一个值,这个值是$f(x(t_{i-1}),y(t_{i-1}),z(t_{i-1}))$.根据定义易得
$$
\int_lf(x,y,z)ds=\lim_{n\to\infty}\sum_{i=1}^nf(x(t_{i-1}),y(t_{i-1}),z(t_{i-1}))\Delta S_{i-1}.
$$
其中 $\Delta S_{i-1}$ 是第 $i$ 段时间段走过的弧长,根据微分中值定理,易得
$$
\frac{\Delta S_{i-1}}{\frac{T-t_0}{n}}=\sqrt{(x_{t_0+i\lambda\frac{T-t_0}{n}}'^2+y_{t_0+i\lambda\frac{T-t_0}{n}}'^2+z_{t_0+i\lambda\frac{T-t_0}{n}}'^2)},0<\lambda<1.
$$
结合导函数的连续性,可得
$$
\Delta S_{i-1}=\sqrt{(x_{t_0+i\frac{T-t_0}{n}}'^2+y_{t_0+i\frac{T-t_0}{n}}'^2+z_{t_0+i\frac{T-t_0}{n}}'^2)}\frac{T-t_0}{n}+o(\frac{T-t_0}{n}).
$$

将其代入上上式,得到
\begin{align*}
\int_lf(x,y,z)ds&=\lim_{n\to\infty}\sum_{i=1}^nf(x(t_{i-q}),y(t_{i-1}),z(t_{i-1}))\sqrt{(x_{t_0+i\frac{T-t_0}{n}}'^2+y_{t_0+i\frac{T-t_0}{n}}'^2+z_{t_0+i\frac{T-t_0}{n}}'^2)}\frac{T-t_0}{n}\\&+\lim_{n\to\infty}no(\frac{1}{n})\\&=\lim_{n\to\infty}\sum_{i=1}^nf(x(t_{i-q}),y(t_{i-1}),z(t_{i-1}))\sqrt{(x_{t_0+i\frac{T-t_0}{n}}'^2+y_{t_0+i\frac{T-t_0}{n}}'^2+z_{t_0+i\frac{T-t_0}{n}}'^2)}\frac{T-t_0}{n}\\&=\int_{t_0}^Tf[(x(t),y(t),z(t)]\sqrt{x_t'^2+y_t'^2+z_t'^2}dt.
\end{align*}
\end{proof}
% ----------------------------------------------------------------------------------------
% BIBLIOGRAPHY
% ----------------------------------------------------------------------------------------
  
\bibliographystyle{unsrt}
  
\bibliography{sample}
  
% ----------------------------------------------------------------------------------------
\end{document}